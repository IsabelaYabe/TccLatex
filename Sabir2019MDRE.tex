\documentclass[12pt,a4paper]{article}
\usepackage[utf8]{inputenc}
\usepackage[T1]{fontenc}

% Idioma e tipografia
\usepackage[brazil]{babel}
\usepackage{csquotes}
\usepackage{lmodern}
\usepackage{microtype}

% Layout e recursos básicos
\usepackage[left=3cm,right=3cm,top=3cm,bottom=3cm]{geometry}
\usepackage{graphicx}
\usepackage{hyperref}
\usepackage{bookmark}

% Matemática e teoremas (essencial só se você usa)
\usepackage{amsmath,amssymb,amsthm}

% Tabelas em paisagem e colunas flexíveis
% \usepackage{pdflscape}
\usepackage{tabularx}
\usepackage{booktabs}
\usepackage{ragged2e}
\usepackage{array} % para \newcolumntype
\newcolumntype{Y}{>{\RaggedRight\arraybackslash}X}
\usepackage{rotating} % para sidewaystable/sideways


% Bibliografia ABNT numerada
\usepackage[
  backend=biber,
  style=abnt,          % estilo bibliográfico
  sorting=none,
  giveninits=true,
  uniquename=false, 
  doi=false,
  isbn=false,
  url=false,
  language=brazil,
  scbib,
  ittitles,
  justify
]{biblatex}
\addbibresource{refs.bib}

% ======= PADRONIZAÇÃO PARA A TABELA MDRE =======

% Coluna flexível "Y" (se ainda não tiver)
% \usepackage{tabularx,booktabs,ragged2e,array}
% \newcolumntype{Y}{>{\RaggedRight\arraybackslash}X}

% 1) Vocabulário controlado (sempre em SMALL CAPS):
\newcommand{\Static}{\textsc{Estático}}
\newcommand{\Dynamic}{\textsc{Dinâmico}}
\newcommand{\Hybrid}{\textsc{Híbrido}}
\newcommand{\Comp}{\textsc{Compreensão}}
\newcommand{\Redoc}{\textsc{Redocumentação}}
\newcommand{\Mig}{\textsc{Migração}}
\newcommand{\Quali}{\textsc{Qualidade}}

% 2) Macros para setas e encadeamentos:
\newcommand{\ctoa}{\(\text{Código} \rightarrow \text{AST}\)}
\newcommand{\atoxi}{\(\text{AST} \rightarrow \text{IM}\)}   % IM = modelo intermediário
\newcommand{\imtoxml}{\(\text{IM} \rightarrow \text{XML}\)}
\newcommand{\imtomdl}{\(\text{IM} \rightarrow \text{UML}\)}
\newcommand{\tmtomdl}{\(\text{T2M/M2M} \rightarrow \text{UML}\)}
\newcommand{\xtoSeq}{\(\rightarrow \text{UML Sequência}\)}
\newcommand{\xtoClass}{\(\rightarrow \text{UML Classe}\)}
\newcommand{\xtoAct}{\(\rightarrow \text{UML Atividade}\)}

% 3) Abreviações de ferramentas (consistentes):
\newcommand{\EMF}{Eclipse/EMF}
\newcommand{\UMLtwo}{UML2}
\newcommand{\PlantUML}{PlantUML}
\newcommand{\JavaParser}{JavaParser}

% 4) Formato da célula “Aspecto”: Técnica ; Objetivo(s)
%    Ex.: \Static; \Comp/\Redoc (estrutura + comportamento)

% 5) Formato da célula “Técnica/Transformação”:
%    Use sempre cadeia com “→”, negrite elementos-chaves e padronize nomes.
%    Ex.: Código → AST → \textbf{IM(XML)} → T2M/M2M → \textbf{UML2}
%
% 6) Formato da célula “Validação”:
%    [tipo de evidência; dataset/projetos; métrica(s) ou avaliação; nota curta]
%    Ex.: OSS (9 projetos, 2640 classes); AUC=0.73; custo de rótulo 10%

\begin{document}

\begin{titlepage}
    \begin{center}
        \vspace*{0cm}
        
            \includegraphics[width=0.5\textwidth]{Images/Logo_FGV.png} 
            
        \vspace{1.5cm}
        \large
        
        Ciência de Dados e I.A.\\
        Escola de Matemática Aplicada\\
        Fundação Getúlio Vargas\\

        \vspace{1cm}  
    
        \Large
        Engenharia de Requisitos
            
        \vspace{2cm}
        
        \vspace{0.25cm}

        \Huge \textbf{TCC} \\ 
        \vspace{0.5cm}
        \huge \textbf{A Model Driven Reverse Engineering Framework for Generating High Level UML Models From Java Source Code}
        \vspace{3.6cm}
        
        \large
                Aluno: Isabela Yabe\\
                Orientador: Rafael de Pinho André\\
                Escola de Matemática Aplicada, FGV/EMAp \\
                Rio de Janeiro - RJ.
        \vfill
            
        \vspace{0.8cm}  
        
        Rio de Janeiro, 2025
            
    \end{center}
\end{titlepage}
\newpage
\pagenumbering{roman}
% \tableofcontents

\newpage
\pagenumbering{arabic}

\section{Revisão literária}
Artigo revisado \textcite{Sabir2019MDRE}:

A revisão tem o objetivo de compreender o estado da arte das abordagens de engenharia reversa que partem de código-fonte e produzem artefatos de alto nível, como diagramas UML. Para garantir uma análise sistemática e comparável entre diferentes propostas, foram definidas perguntas de pesquisa (\textit{Research Questions — RQs}) que orientam a coleta e síntese dos dados extraídos dos estudos selecionados.

\begin{itemize}
  \item \textbf{RQ1.} Em quais linguagens e domínios as abordagens que partem de código-fonte foram aplicadas?
  \item \textbf{RQ2.} Quais modelos/artefatos de alto nível são gerados?
  \item \textbf{RQ3.} Qual aspecto é privilegiado (estático, dinâmico, híbrido) e com qual objetivo (compreensão, redocumentação, migração, qualidade)?
  \item \textbf{RQ4.} Quais técnicas e transformações viabilizam a passagem do código para o modelo de alto nível?
  \item \textbf{RQ5.} Quais ferramentas/frameworks são utilizados?
  \item \textbf{RQ6.} Como as abordagens são validadas e com que qualidade prática?
\end{itemize}

\section{RQ1. Em quais linguagens e domínios as abordagens que partem de código-fonte foram aplicadas?}
Neste trabalho, o domínio é Java, a abordagem extrai e transforma código-fonte Java em modelos estruturais e comportamentais UML.

\section{RQ2. Quais modelos/artefatos de alto nível são gerados?}

São gerados os modelos de alto nível finais Class Diagram e Activity Diagram, o resultado é entregue como um UML package com esses modelos.
Também são gerados artefatos de suporte, o Intermediate Model (XML) viabiliza as transformações T2M/M2M e cada Function Behavior serve de ponte para construir o activity por operação.

\section{RQ3. Qual aspecto é privilegiado (estático, dinâmico, híbrido) e com qual objetivo (compreensão, redocumentação, migração, qualidade)?}
O aspecto privilegiado é o estático (parsing do código -> AST -> modelo intermediário -> UML), todo o pipeline é T2M estático (texto -> modelo). Com objetivo principal de compreensão e redocumentação.

\section{RQ4. Quais técnicas e transformações viabilizam a passagem do código para o modelo de alto nível?}
A passagem do código para modelo UML é viabilizada por um pipeline T2M/M2M em duas fases, (i) descoberta do modelo intermediário (IMD), (ii) gerador do modelo UML.

Na primeira fase o código é parseado para um modelo de fácil compreensão (JavaParser) e transformado numa AST com apenas informações relevantes, então informações de estrutura e comportamento são extraídos da AST gerando o modelo intermediário (IM) em XML (UML2-EMF).

Na segunda fase são aplicadas regras de mapeamento do IM para artefatos UML, tanto mapeamento estrutural quando comportamental.

\section{RQ5. Quais ferramentas/frameworks são utilizados?}

A implementação do \textit{Src2MoF} apoia-se em três pilares tecnológicos: (i) \textbf{Eclipse} e a plataforma \textbf{UML2/EMF} para implementar o gerador de modelos e as transformações \emph{Text-to-Model}/\emph{Model-to-Model}; (ii) o \textbf{JavaParser}, integrado ao \emph{Intermediate Model Discoverer} (IMD), para analisar o código Java e construir uma AST refinada, a partir da qual é derivado um \emph{Intermediate Model} (IM) textual em XML; e (iii) o ecossistema \textbf{UML2} para materializar os modelos UML de alto nível (diagramas de classes e de atividades) a partir do IM.

Especificamente, o \emph{model generator} é \enquote{carried out using Eclipse tool} e suas regras de transformação são \enquote{written and implemented in UML2 platform}, acionando dois \emph{templates} (\emph{Class Diagram Creator} e \emph{Activity Diagram Creator}) \textcite{Sabir2019MDRE}. Já o IMD \enquote{uses an open source tool \textquote{Java Parser}} para produzir a AST e, a partir dela, extrair estrutura e comportamento para um IM em XML \textcite{Sabir2019MDRE}. Para a comparação manual na avaliação, os autores mencionam Papyrus, StarUML, Rational Rose e o editor UML do Eclipse; tais ferramentas não integram o \emph{pipeline} automático do \textit{Src2MoF}.

\noindent\textit{Síntese (RQ5).} \emph{Eclipse + UML2/EMF} (T2M/M2M); \emph{JavaParser} (código $\rightarrow$ AST $\rightarrow$ IM/XML); \emph{Eclipse, Papyrus, StarUML, Rational Rose} (apoio à validação).

\section{RQ6. Como as abordagens são validadas e com que qualidade prática?}

A validação do \textit{Src2MoF} baseia-se na comparação entre os modelos gerados automaticamente e modelos elaborados manualmente por especialistas, configurando uma evidência empírica e qualitativa, sem reporte de métricas quantitativas de acurácia ou de tempo de execução. Nessa configuração, engenheiros de software primeiro constroem, com ferramentas de modelagem (Eclipse, Papyrus, StarUML e Rational Rose), diagramas de classes e de atividades a partir do mesmo código que, em seguida, é processado pelo \textit{Src2MoF}; a análise confronta as duas saídas. O protocolo é aplicado a cinco estudos de caso de referência, dos quais dois são detalhados no artigo: \textit{Automated Teller Machine (ATM)} e \textit{Amadeus Hospitality}. Permitindo observar recorrências e limites no comportamento do framework.

Nos resultados, os diagramas de classes produzidos refletem, de forma consistente, atributos, operações e relacionamentos observáveis no código, sugerindo boa correspondência estrutural. Do ponto de vista comportamental, o framework deriva uma \texttt{Activity} para cada método, traduzindo o corpo da operação em \texttt{OpaqueAction} e mobilizando \texttt{CallOperationAction} e \texttt{CreateObjectAction}, além de nós iniciais e finais, nós condicionais e fluxos de controle e de objeto; tal mapeamento sustenta a reivindicação de que estrutura e comportamento são gerados de maneira conjunta a partir do código.

Contudo, a qualidade prática reportada permanece no domínio da conformidade percebida: não há estimativas de precisão/recall, nem análise de concordância entre avaliadores, escalabilidade ou custo temporal. Além disso, a implementação divulgada restringe-se à linguagem Java, foca \textit{Activity} como diagrama comportamental e não cobre certos construtos (por exemplo, nós \textit{merge} e \textit{fork}).

\begin{table}[h!]
\centering
\scriptsize % ou \footnotesize, se quiser mais compacta
\setlength{\tabcolsep}{4pt} % espaçamento horizontal entre colunas
\renewcommand{\arraystretch}{1.2} % espaçamento vertical entre linhas

\begin{tabularx}{\textwidth}{Y Y Y Y Y Y Y}
\toprule
\textbf{Autores / Referência} &
\textbf{Linguagem / Domínio} &
\textbf{Modelo Gerado} &
\textbf{Aspecto} &
\textbf{Técnica / Transformação} &
\textbf{Ferramenta / Framework} &
\textbf{Validação / Estudo de Caso} \\
\midrule

\textcite{Sabir2019MDRE} &
Java (sistemas legados orientados a objetos) &
UML \emph{Class Diagram} + \emph{Activity Diagram} (em pacote UML) &
Estático; objetivo: compreensão/redocumentação &
T2M/M2M em duas fases: Parser$\rightarrow$AST$\rightarrow$IM(XML) e mapeamentos IM$\rightarrow$UML (classe/atividade) &
Eclipse + UML2/EMF; JavaParser (IMD); (Papyrus/StarUML/Rational Rose na validação manual) &
Comparação especialista (modelos manuais vs. gerados), 5 estudos de caso; ATM e Amadeus descritos \\ 

\bottomrule
\end{tabularx}
\caption{Resumo das abordagens}
\label{tab:mdre6}
\end{table}

\newpage

\printbibliography

\end{document}
