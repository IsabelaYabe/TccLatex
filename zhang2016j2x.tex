\documentclass[12pt,a4paper]{article}
\usepackage[utf8]{inputenc}
\usepackage[T1]{fontenc}

% Idioma e tipografia
\usepackage[brazil]{babel}
\usepackage{csquotes}
\usepackage{lmodern}
\usepackage{microtype}

% Layout e recursos básicos
\usepackage[left=3cm,right=3cm,top=3cm,bottom=3cm]{geometry}
\usepackage{graphicx}
\usepackage{hyperref}
\usepackage{bookmark}

% Matemática e teoremas (essencial só se você usa)
\usepackage{amsmath,amssymb,amsthm}

% Tabelas em paisagem e colunas flexíveis
% \usepackage{pdflscape}
\usepackage{tabularx}
\usepackage{booktabs}
\usepackage{ragged2e}
\usepackage{array} % para \newcolumntype
\newcolumntype{Y}{>{\RaggedRight\arraybackslash}X}
\usepackage{rotating} % para sidewaystable/sideways


% Bibliografia ABNT numerada
\usepackage[
  backend=biber,
  style=abnt,          % estilo bibliográfico
  sorting=none,
  giveninits=true,
  uniquename=false, 
  doi=false,
  isbn=false,
  url=false,
  language=brazil,
  scbib,
  ittitles,
  justify
]{biblatex}
\addbibresource{refs.bib}

% ======= PADRONIZAÇÃO PARA A TABELA MDRE =======

% Coluna flexível "Y" (se ainda não tiver)
% \usepackage{tabularx,booktabs,ragged2e,array}
% \newcolumntype{Y}{>{\RaggedRight\arraybackslash}X}

% 1) Vocabulário controlado (sempre em SMALL CAPS):
\newcommand{\Static}{\textsc{Estático}}
\newcommand{\Dynamic}{\textsc{Dinâmico}}
\newcommand{\Hybrid}{\textsc{Híbrido}}
\newcommand{\Comp}{\textsc{Compreensão}}
\newcommand{\Redoc}{\textsc{Redocumentação}}
\newcommand{\Mig}{\textsc{Migração}}
\newcommand{\Quali}{\textsc{Qualidade}}

% 2) Macros para setas e encadeamentos:
\newcommand{\ctoa}{\(\text{Código} \rightarrow \text{AST}\)}
\newcommand{\atoxi}{\(\text{AST} \rightarrow \text{IM}\)}   % IM = modelo intermediário
\newcommand{\imtoxml}{\(\text{IM} \rightarrow \text{XML}\)}
\newcommand{\imtomdl}{\(\text{IM} \rightarrow \text{UML}\)}
\newcommand{\tmtomdl}{\(\text{T2M/M2M} \rightarrow \text{UML}\)}
\newcommand{\xtoSeq}{\(\rightarrow \text{UML Sequência}\)}
\newcommand{\xtoClass}{\(\rightarrow \text{UML Classe}\)}
\newcommand{\xtoAct}{\(\rightarrow \text{UML Atividade}\)}

% 3) Abreviações de ferramentas (consistentes):
\newcommand{\EMF}{Eclipse/EMF}
\newcommand{\UMLtwo}{UML2}
\newcommand{\PlantUML}{PlantUML}
\newcommand{\JavaParser}{JavaParser}

% 4) Formato da célula “Aspecto”: Técnica ; Objetivo(s)
%    Ex.: \Static; \Comp/\Redoc (estrutura + comportamento)

% 5) Formato da célula “Técnica/Transformação”:
%    Use sempre cadeia com “→”, negrite elementos-chaves e padronize nomes.
%    Ex.: Código → AST → \textbf{IM(XML)} → T2M/M2M → \textbf{UML2}
%
% 6) Formato da célula “Validação”:
%    [tipo de evidência; dataset/projetos; métrica(s) ou avaliação; nota curta]
%    Ex.: OSS (9 projetos, 2640 classes); AUC=0.73; custo de rótulo 10%

\begin{document}

\begin{titlepage}
    \begin{center}
        \vspace*{0cm}
        
            \includegraphics[width=0.5\textwidth]{Images/Logo_FGV.png} 
            
        \vspace{1.5cm}
        \large
        
        Ciência de Dados e I.A.\\
        Escola de Matemática Aplicada\\
        Fundação Getúlio Vargas\\

        \vspace{1cm}  
    
        \Large
        Engenharia de Requisitos
            
        \vspace{2cm}
        
        \vspace{0.25cm}

        \Huge \textbf{TCC} \\ 
        \vspace{0.5cm}
        \huge \textbf{An Approach for Extracting UML Diagram from Object-Oriented Program Based on J2X}

        \vspace{3.6cm}
        
        \large
                Aluno: Isabela Yabe\\
                Orientador: Rafael de Pinho André\\
                Escola de Matemática Aplicada, FGV/EMAp \\
                Rio de Janeiro - RJ.
        \vfill
            
        \vspace{0.8cm}  
        
        Rio de Janeiro, 2025
            
    \end{center}
\end{titlepage}
\newpage
\pagenumbering{roman}
% \tableofcontents

\newpage
\pagenumbering{arabic}

\section{Revisão literária}
Artigo revisado \textcite{zhang2016j2x}:

A revisão tem o objetivo de compreender o estado da arte das abordagens de engenharia reversa que partem de código-fonte e produzem artefatos de alto nível, como diagramas UML. Para garantir uma análise sistemática e comparável entre diferentes propostas, foram definidas perguntas de pesquisa (\textit{Research Questions — RQs}) que orientam a coleta e síntese dos dados extraídos dos estudos selecionados.

\begin{itemize}
  \item \textbf{RQ1.} Em quais linguagens e domínios as abordagens que partem de código-fonte foram aplicadas?
  \item \textbf{RQ2.} Quais modelos/artefatos de alto nível são gerados?
  \item \textbf{RQ3.} Qual aspecto é privilegiado (estático, dinâmico, híbrido) e com qual objetivo (compreensão, redocumentação, migração, qualidade)?
  \item \textbf{RQ4.} Quais técnicas e transformações viabilizam a passagem do código para o modelo de alto nível?
  \item \textbf{RQ5.} Quais ferramentas/frameworks são utilizados?
  \item \textbf{RQ6.} Como as abordagens são validadas e com que qualidade prática?
\end{itemize}

\section{RQ1. Em quais linguagens e domínios as abordagens que partem de código-fonte foram aplicadas?}

No artigo analisado, tanto a implementação da ferramenta quanto a avaliação empírica concentram-se na linguagem Java. Em termos de domínio, os autores testam a abordagem em pequenos sistemas orientados a objetos de propósito geral, típicos de benchmarks didáticos e bibliotecas: um jogo (\textit{Minesweeper}), um blog (\textit{Blog}), um sistema de folha de pagamento (\textit{PayrollSys}), uma e-library (\textit{eLib}) e uma biblioteca de algoritmos (\textit{myAlgsLib}). Esse conjunto de casos sinaliza que a aplicação prática recai sobre programas de escopo reduzido e natureza genérica, sem amarração a um domínio específico de negócio.

Por fim, embora todo o exercício experimental esteja restrito ao ecossistema Java, os autores afirmam que o uso do intermediário J2X foi concebido para “isolar diferenças de linguagem” e, assim, permitir a reutilização do mesmo método de \textit{reverse} em outras linguagens, uma generalidade pretendida no plano conceitual, mas não demonstrada empiricamente no estudo.

Síntese de RQ1: 
Linguagem aplicada: Java (implementação e avaliação).

Domínio dos estudos: pequenos sistemas OO genéricos (jogo, blog, sistema de folha de pagamento, biblioteca de algoritmos, e-library).

Escopo pretendido: método projetado para ser estendido a outras linguagens via J2X (afirmação conceitual, não demonstrada empiricamente no artigo).


\section{RQ2. Quais modelos/artefatos de alto nível são gerados?}

O método gera dois artefatos UML de alto nível: (i) diagrama de classes (estrutura) e (ii) diagrama de sequência (comportamento de interação). 

Operacionalmente, o framework (a) “first transforms source code to J2X representation”, (b) constrói o diagrama de classes com base nas informações extraídas, e (c) combina OFG e CFG para “construct UML sequence diagram”. Esses elementos (J2X, OFG, CFG) são artefatos intermediários.

Cadeia que conduz a esses modelos (código → J2X → OFG/CFG → UML) e menções a “construct UML class diagram” e “construct UML sequence diagram”: Abstract e Approach Overview.

Síntese para o texto do TCC (sugerido):
“O método entrega, como artefatos de alto nível, o UML Class Diagram e o UML Sequence Diagram. A geração é suportada por uma cadeia estática que transforma o código em J2X e analisa OFG/CFG, culminando na construção dos diagramas (‘construct UML class diagram’; ‘construct UML sequence diagram’).”

\section{RQ3. Qual aspecto é privilegiado (estático, dinâmico, híbrido) e com qual objetivo?}

A abordagem é de \Static, pois opera exclusivamente sobre o código-fonte, sem instrumentação nem execução. O objetivo principal é \Comp\ e \Redoc\ de sistemas orientados a objetos, por meio da elevação do nível de abstração: do código para modelos UML. Em particular, gera-se \textit{UML Classe} (estrutura) e \textit{UML Sequência} (interação) a partir de informações derivadas estaticamente do código, o que apoia entendimento, manutenção e documentação do sistema.

Em síntese, trata-se de uma análise \Static\ que extrai tanto a estrutura (classes e relacionamentos) quanto a interação (mensagens em sequência) com base em \textit{artefatos derivados do próprio código}, preparando o terreno para as transformações discutidas na RQ4.


\paragraph{Extensão proposta (minha versão).}

Os autores reconhecem a possibilidade de um futuro híbrido (estático + dinâmico) para aumentar a acurácia. Como extensão, propomos um \Hybrid\ que combine a análise estática com uma \textit{análise de intenção} (o que o software está intencionado a fazer), inferida de indícios semânticos do código (nomes, contratos, comentários) e correlacionada a casos de uso. A hipótese é que a integração entre (\Static) e intenção de alto nível melhore a precisão na identificação de interações relevantes e reduza ambiguidades na geração de UML Sequência, mantendo o baixo custo de coleta (sem execução).

\section{RQ4. Quais técnicas e transformações viabilizam a passagem do código para o modelo de alto nível?}

\emph{(1) Código $\rightarrow$ AST $\rightarrow$ J2X (IR).} O método realiza o \textit{parsing} do código, visita a AST e anota semântica em uma linguagem intermediária XML (\textit{J2X}) que padroniza elementos estruturais e “isola diferenças de linguagem”. 
\textit{Efeito prático:} a J2X viabiliza armazenamento/troca de informações e estabiliza as etapas seguintes sem depender novamente do código-fonte.

\emph{(2) J2X $\rightarrow$ metadados + mapeamentos para \textit{UML Classe}.} A partir do J2X, extrai-se a estrutura básica (classes, interfaces, métodos, campos) e mapeiam-se quatro relações estruturais para o diagrama de classes: generalização, implementação, associação e dependencia.
\textit{Efeito prático:} torna explícitas hierarquias (generalização/realização) e acoplamentos (associação/dependência), úteis a \Comp/\Redoc.

\emph{(3) Refinamento de relações com OFG (sentenças simplificadas $\rightarrow$ OFG).} Para aumentar a precisão de associação/dependência, a representação J2X é reduzida a “sentenças simplificadas” e então usada para construir um \textit{Object Flow Graph} (OFG) que rastreia a propagação de objetos.
\textit{Efeito prático:} reduz falsos positivos/negativos como polimorfismo nas relações estruturais ao confirmar tipos e fluxos reais de objetos.

\emph{(4) J2X $\rightarrow$ CFG $\rightarrow$ (OFG + CFG) $\rightarrow$ \textit{UML Sequência}.} Para o diagrama de sequência, o método modela chamadas e estruturas de controle em um \textit{Control Flow Graph} (CFG) e o combina com o OFG a fim de identificar objetos participantes, mensagens e condições/loops.
\textit{Efeito prático:} mesmo sem execução, a combinação OFG+CFG preserva ordem/condição das interações, permitindo gerar \textit{UML Sequência} com \emph{lifelines}, \emph{messages} e \emph{fragments} (\textsf{alt}/\textsf{opt}/\textsf{loop}).

\medskip
\noindent\textbf{Síntese de RQ4:} 
\begin{itemize}
  \item \textbf{Intermediário:} Código $\rightarrow$ AST $\rightarrow$ \textbf{J2X (XML)}.
  \item \textbf{Classe:} metadados + mapeamentos diretos (generalização, realização, associação, dependência).
  \item \textbf{Precisão de relações:} sentenças simplificadas $\rightarrow$ \textbf{OFG} (fluxo de objetos).
  \item \textbf{Sequência:} \textbf{CFG} (fluxo de controle) + \textbf{OFG} $\Rightarrow$ mensagens/condições sem execução.
\end{itemize}


\section{RQ5. Quais ferramentas/frameworks são utilizados?}

Em alinhamento com o pipeline técnico descrito na RQ4, o ecossistema de suporte compreende: (i) a ferramenta de engenharia reversa que orquestra a extração e geração de diagramas; (ii) o gerador de \textit{parser} para obtenção da AST; (iii) a infraestrutura de processamento XML sobre a linguagem intermediária; e (iv) a própria linguagem intermediária (IR), que, embora não seja uma ferramenta, é o artefato central de representação.

\paragraph{Ferramenta de engenharia reversa (J2UML).}
Os autores implementam a ferramenta que materializa o processo de \textit{reverse}.
\textit{Papel no pipeline:} integra a leitura do J2X, aplica mapeamentos para \textit{UML Classe} e combina OFG/CFG para \textit{UML Sequência}, entregando os artefatos de alto nível úteis a \Comp/\Redoc.

\paragraph{Gerador de \textit{parser} (JavaCC).}
O \textit{frontend} do processo usa um gerador de analisadores para construir a AST:
\foreignblockquote{english}[\textcite{zhang2016j2x}]{``[The parser is] automatic generated by JavaCC.''}
\textit{Papel no pipeline:} viabiliza \emph{Código} $\rightarrow$ \emph{AST}, primeiro passo para rotular semântica no J2X.

\paragraph{Infraestrutura de processamento XML (DOM4J e equivalentes).}
Para manipular a representação intermediária, os autores recorrem a bibliotecas maduras
\textit{Papel no pipeline:} permite consultar/transformar o \textbf{J2X (XML)} com eficiência e baixo acoplamento ao \textit{parser}.

\paragraph{Linguagem intermediária (J2X, DTD/XML).}
Document Type Definition (DTD) define a estrutura do XML usado como IR.
A J2X é a IR que armazena semântica e estrutura do programa.
A especificação em DTD padroniza elementos e atributos, sustentando a extração estruturada (\emph{Código} $\rightarrow$ \emph{AST} $\rightarrow$ \textbf{J2X}).

\paragraph{Ambiente de execução (contexto experimental).}
Para reprodutibilidade, os autores reportam o contexto de execução:
\foreignblockquote{english}[\textcite{zhang2016j2x}]{``The tool runs on a PC with Windows [...] 32bit, 3G Memory and 2.93Ghz Intel Core 2 Duo CPU.''}
\textit{Observação:} trata-se de caracterização do \emph{setup} usado nos experimentos, não de requisito da abordagem.

\medskip
\noindent\textbf{Síntese de RQ5:}
\begin{itemize}
  \item \textbf{Ferramenta central:} J2UML (orquestra extração e geração de UML).
  \item \textbf{Parser/AST:} gerado com \textbf{JavaCC}.
  \item \textbf{Processamento da IR:} bibliotecas XML (e.g., \textbf{DOM4J}).
  \item \textbf{Representação:} \textbf{J2X (DTD/XML)} como linguagem intermediária.
  \item \textbf{Ambiente de execução:} especificado para reprodutibilidade dos resultados.
\end{itemize}


\section{RQ6. Como as abordagens são validadas e com que qualidade prática?}

Fonte no artigo e como extraí: toda a evidência de validação está na seção Experiment and Evaluation e na Tabela 4 do paper; ali os autores descrevem o procedimento, os estudos de caso, as métricas e interpretam os resultados. Extraí os dados diretamente desse trecho.

Os autores implementam a ferramenta J2UML e a avaliam em um conjunto de casos de teste de pequeno porte, medindo desempenho (tempo de execução) e exatidão na extração do diagrama de classes. Os dados observados são: número de classes, tempo, e contagens extraídas pela ferramenta.

A “acurácia” é calculada por comparação manual com um conhecimento de referência derivado do código dos cinco casos, tanto para classes quanto para relações.

Interpretação dos autores.
A extração de classes é mais precisa que a de relações; os autores consideram os erros “aceitáveis” e afirmam que sua análise de fluxo de objetos ajuda a obter relações mais acuradas. 

A validação foi conduzida por estudo experimental com cinco sistemas OO de pequeno porte, medindo tempo e acurácia da extração do diagrama de classes. A acurácia foi definida como a razão entre itens corretamente extraídos e o ground truth manual (classes e relações). Os resultados indicam acurácia de 96,4–100\% para classes e 65,0–90,4\% para relações, com desempenho considerado aceitável, variando conforme a complexidade das dependências.

\medskip
\noindent\textbf{Síntese de RQ6:}
\begin{itemize}
  \item \textbf{Desenho}: estudo experimental com 5 sistemas OO pequenos; comparação ferramenta vs. \textit{ground truth} manual.
  \item \textbf{Métricas}: tempo de execução; acurácia de classes e relações (definições explícitas).
  \item \textbf{Resultados}: classes \(\approx 96{,}4\%\)–\(100\%\); relações \(\approx 65{,}0\%\)–\(90{,}4\%\); tempos em ms.
  \item \textbf{Qualidade prática}: útil para \Comp/\Redoc\ com custo baixo; OFG melhora relações.
  \item \textbf{Limitações}: casos didáticos; pouca evidência para sistemas grandes; \textit{ground truth} manual; sem análise estatística.
\end{itemize}

\begin{table}[h!]
\centering
\scriptsize % ou \footnotesize, se quiser mais compacta
\setlength{\tabcolsep}{4pt} % espaçamento horizontal entre colunas
\renewcommand{\arraystretch}{1.2} % espaçamento vertical entre linhas

\begin{tabularx}{\textwidth}{Y Y Y Y Y Y Y}
\toprule
\textbf{Autores / Referência} &
\textbf{Linguagem / Domínio} &
\textbf{Modelo Gerado} &
\textbf{Aspecto} &
\textbf{Técnica / Transformação} &
\textbf{Ferramenta / Framework} &
\textbf{Validação / Estudo de Caso} \\
\midrule

\midrule
\textcite{zhang2016j2x} &
Java; pequenos sistemas OO (eLib, Minesweeper, Blog, PayrollSys, myAlgsLib) &
UML Classe; UML Sequência &
Estático — compreensão, manutenção/redocumentação &
Código→AST→J2X; mapeamentos (gen./impl./assoc./dep.); sentenças simplif.→OFG; CFG+OFG→Sequência &
J2UML; JavaCC; Dom4j; J2X (DTD/XML) &
5 casos pequenos; acurácia: classes 96,4–100\%; relações 65,0–90,4\% \\


\bottomrule
\end{tabularx}
\caption{Resumo das abordagens}
\label{tab:mdre6}
\end{table}

\newpage

\printbibliography

\end{document}
