\documentclass[12pt,a4paper]{article}
\usepackage[utf8]{inputenc}
\usepackage[T1]{fontenc}

% Idioma e tipografia
\usepackage[brazil]{babel}
\usepackage{csquotes}
\usepackage{lmodern}
\usepackage{microtype}

% Layout e recursos básicos
\usepackage[left=3cm,right=3cm,top=3cm,bottom=3cm]{geometry}
\usepackage{graphicx}
\usepackage{hyperref}
\usepackage{bookmark}

% Matemática e teoremas (essencial só se você usa)
\usepackage{amsmath,amssymb,amsthm}

% Tabelas em paisagem e colunas flexíveis
% \usepackage{pdflscape}
\usepackage{tabularx}
\usepackage{booktabs}
\usepackage{ragged2e}
\usepackage{array} % para \newcolumntype
\newcolumntype{Y}{>{\RaggedRight\arraybackslash}X}
\usepackage{rotating} % para sidewaystable/sideways


% Bibliografia ABNT numerada
\usepackage[
  backend=biber,
  style=abnt,          % estilo bibliográfico
  sorting=none,
  giveninits=true,
  uniquename=false, 
  doi=false,
  isbn=false,
  url=false,
  language=brazil,
  scbib,
  ittitles,
  justify
]{biblatex}
\addbibresource{refs.bib}

% ======= PADRONIZAÇÃO PARA A TABELA MDRE =======

% Coluna flexível "Y" (se ainda não tiver)
% \usepackage{tabularx,booktabs,ragged2e,array}
% \newcolumntype{Y}{>{\RaggedRight\arraybackslash}X}

% 1) Vocabulário controlado (sempre em SMALL CAPS):
\newcommand{\Static}{\textsc{Estático}}
\newcommand{\Dynamic}{\textsc{Dinâmico}}
\newcommand{\Hybrid}{\textsc{Híbrido}}
\newcommand{\Comp}{\textsc{Compreensão}}
\newcommand{\Redoc}{\textsc{Redocumentação}}
\newcommand{\Mig}{\textsc{Migração}}
\newcommand{\Quali}{\textsc{Qualidade}}

% 2) Macros para setas e encadeamentos:
\newcommand{\ctoa}{\(\text{Código} \rightarrow \text{AST}\)}
\newcommand{\atoxi}{\(\text{AST} \rightarrow \text{IM}\)}   % IM = modelo intermediário
\newcommand{\imtoxml}{\(\text{IM} \rightarrow \text{XML}\)}
\newcommand{\imtomdl}{\(\text{IM} \rightarrow \text{UML}\)}
\newcommand{\tmtomdl}{\(\text{T2M/M2M} \rightarrow \text{UML}\)}
\newcommand{\xtoSeq}{\(\rightarrow \text{UML Sequência}\)}
\newcommand{\xtoClass}{\(\rightarrow \text{UML Classe}\)}
\newcommand{\xtoAct}{\(\rightarrow \text{UML Atividade}\)}

% 3) Abreviações de ferramentas (consistentes):
\newcommand{\EMF}{Eclipse/EMF}
\newcommand{\UMLtwo}{UML2}
\newcommand{\PlantUML}{PlantUML}
\newcommand{\JavaParser}{JavaParser}

% 4) Formato da célula “Aspecto”: Técnica ; Objetivo(s)
%    Ex.: \Static; \Comp/\Redoc (estrutura + comportamento)

% 5) Formato da célula “Técnica/Transformação”:
%    Use sempre cadeia com “→”, negrite elementos-chaves e padronize nomes.
%    Ex.: Código → AST → \textbf{IM(XML)} → T2M/M2M → \textbf{UML2}
%
% 6) Formato da célula “Validação”:
%    [tipo de evidência; dataset/projetos; métrica(s) ou avaliação; nota curta]
%    Ex.: OSS (9 projetos, 2640 classes); AUC=0.73; custo de rótulo 10%

\begin{document}

\begin{titlepage}
    \begin{center}
        \vspace*{0cm}
        
            \includegraphics[width=0.5\textwidth]{Images/Logo_FGV.png} 
            
        \vspace{1.5cm}
        \large
        
        Ciência de Dados e I.A.\\
        Escola de Matemática Aplicada\\
        Fundação Getúlio Vargas\\

        \vspace{1cm}  
    
        \Large
        Engenharia de Requisitos
            
        \vspace{2cm}
        
        \vspace{0.25cm}

        \Huge \textbf{TCC} \\ 
        \vspace{0.5cm}
        \huge \textbf{An Approach for Extracting UML Diagram from Object-Oriented Program Based on J2X}

        \vspace{3.6cm}
        
        \large
                Aluno: Isabela Yabe\\
                Orientador: Rafael de Pinho André\\
                Escola de Matemática Aplicada, FGV/EMAp \\
                Rio de Janeiro - RJ.
        \vfill
            
        \vspace{0.8cm}  
        
        Rio de Janeiro, 2025
            
    \end{center}
\end{titlepage}
\newpage
\pagenumbering{roman}
% \tableofcontents

\newpage
\pagenumbering{arabic}

\section{Revisão literária}
Artigo revisado \textcite{zhang2016j2x}:

A revisão tem o objetivo de compreender o estado da arte das abordagens de engenharia reversa que partem de código-fonte e produzem artefatos de alto nível, como diagramas UML. Para garantir uma análise sistemática e comparável entre diferentes propostas, foram definidas perguntas de pesquisa (\textit{Research Questions — RQs}) que orientam a coleta e síntese dos dados extraídos dos estudos selecionados.

\begin{itemize}
  \item \textbf{RQ1.} Em quais linguagens e domínios as abordagens que partem de código-fonte foram aplicadas?
  \item \textbf{RQ2.} Quais modelos/artefatos de alto nível são gerados?
  \item \textbf{RQ3.} Qual aspecto é privilegiado (estático, dinâmico, híbrido) e com qual objetivo (compreensão, redocumentação, migração, qualidade)?
  \item \textbf{RQ4.} Quais técnicas e transformações viabilizam a passagem do código para o modelo de alto nível?
  \item \textbf{RQ5.} Quais ferramentas/frameworks são utilizados?
  \item \textbf{RQ6.} Como as abordagens são validadas e com que qualidade prática?
\end{itemize}

\section{RQ1. Em quais linguagens e domínios as abordagens que partem de código-fonte foram aplicadas?}

A implementação e avaliação concentram-se na linguagem Java. Os autores testam a abordagem em cinco pequenos sistemas orientados a objetos de propósito geral: \textit{Minesweeper}, \textit{Blog}, \textit{PayrollSys}, \textit{eLib} e \textit{myAlgsLib}. Esses casos evidenciam a aplicação em programas de escopo reduzido e natureza genérica, sem vínculo a um domínio específico.

Embora a experimentação esteja restrita ao ecossistema Java, o artigo destaca que o uso do intermediário J2X foi concebido para “isolar diferenças de linguagem” e permitir a extensão do método a outras linguagens.


\section{RQ2. Quais modelos/artefatos de alto nível são gerados?}
O método gera dois artefatos UML de alto nível: o diagrama de classes (estrutura) e o diagrama de sequência (comportamento de interação). O processo ocorre em três etapas: (i) transformação do código em representação J2X, (ii) extração estrutural para o diagrama de classes e (iii) análise de fluxo (OFG e CFG) para construir o diagrama de sequência. Assim, a cadeia de transformação é: \textit{código} $\rightarrow$ \textit{J2X} $\rightarrow$ \textit{OFG/CFG} $\rightarrow$ \textit{UML}.

\section{RQ3. Qual aspecto é privilegiado (estático, dinâmico, híbrido) e com qual objetivo?}
A abordagem é predominantemente \Static, operando sobre o código-fonte sem execução. O objetivo central é a \Comp\ e \Redoc\ de sistemas OO, elevando o nível de abstração por meio da geração de diagramas UML de classes e sequência.

\paragraph{Extensão proposta (minha proposta).}

Os autores reconhecem a possibilidade de um futuro híbrido (estático + dinâmico) para aumentar a acurácia. Como extensão, propomos um \Hybrid\ que combine a análise estática com uma \textit{análise de intenção} (o que o software está intencionado a fazer), inferida de indícios semânticos do código (nomes, contratos, comentários) e correlacionada a casos de uso. A hipótese é que a integração entre (\Static) e intenção de alto nível melhore a precisão na identificação de interações relevantes e reduza ambiguidades na geração de UML Sequência, mantendo o baixo custo de coleta (sem execução).

\section{RQ4. Quais técnicas e transformações viabilizam a passagem do código para o modelo de alto nível?}
O processo segue um pipeline de quatro fases principais:

\begin{enumerate}
  \item \textbf{Código $\rightarrow$ AST $\rightarrow$ J2X (IR):} parsing e anotação semântica do código em linguagem intermediária XML (J2X), que padroniza elementos e isola diferenças de linguagem.
  \item \textbf{J2X $\rightarrow$ metadados $\rightarrow$ UML Classe:} extração de classes, interfaces, métodos e campos, mapeando relações de generalização, implementação, associação e dependência.
  \item \textbf{OFG (Object Flow Graph):} refinamento das relações por rastreamento de fluxos de objetos, reduzindo falsos positivos/negativos e tratando polimorfismo.
  \item \textbf{J2X $\rightarrow$ CFG $\rightarrow$ (OFG + CFG) $\rightarrow$ UML Sequência:} 
Para a geração do diagrama de sequência, o método modela as chamadas de métodos e as estruturas de controle do programa em um \textit{Control Flow Graph} (CFG). Em seguida, esse grafo é combinado ao \textit{Object Flow Graph} (OFG), de modo a associar o fluxo de controle às instâncias de objetos efetivamente envolvidos nas interações. 
Essa integração permite identificar, de forma estática, os objetos participantes, as mensagens trocadas e as condições e repetições (\textsf{alt}/\textsf{opt}/\textsf{loop}) que estruturam o comportamento do sistema. 
\end{enumerate}

\textit{OFG+CFG,efeito prático:} mesmo sem instrumentação ou execução, a combinação preserva a ordem e a lógica das interações entre objetos, possibilitando a geração automática de \textit{UML Sequence Diagrams} completos, contendo \emph{lifelines}, \emph{messages} e \emph{interaction fragments}.

\section{RQ5. Quais ferramentas/frameworks são utilizados?}
O ecossistema técnico inclui:

\begin{itemize}
  \item \textbf{J2UML:} ferramenta desenvolvida pelos autores que orquestra a extração e geração dos diagramas UML.
  \item \textbf{JavaCC:} gerador de analisadores automáticos usado para construir a AST a partir do código-fonte.
  \item \textbf{DOM4J:} biblioteca para manipulação eficiente do XML que representa o modelo intermediário J2X.
  \item \textbf{J2X (DTD/XML):} linguagem intermediária padronizada para estruturar e armazenar informações semânticas do programa.
\end{itemize}

Os experimentos foram realizados em um ambiente Windows 32-bit, com 3GB de memória e CPU Intel Core 2 Duo a 2,93GHz, garantindo reprodutibilidade dos resultados.

\section{RQ6. Como as abordagens são validadas e com que qualidade prática?}

Os autores implementam a ferramenta J2UML e a avaliam em um conjunto de casos de teste de pequeno porte, medindo desempenho (tempo de execução) e exatidão na extração do diagrama de classes. Os dados observados são: número de classes, tempo, e contagens extraídas pela ferramenta.

A “acurácia” é calculada por comparação manual com um conhecimento de referência derivado do código dos cinco casos, tanto para classes quanto para relações.

Interpretação dos autores.
A extração de classes é mais precisa que a de relações; os autores consideram os erros “aceitáveis” e afirmam que sua análise de fluxo de objetos ajuda a obter relações mais acuradas. 

A validação foi conduzida por estudo experimental com cinco sistemas OO de pequeno porte, medindo tempo e acurácia da extração do diagrama de classes. A acurácia foi definida como a razão entre itens corretamente extraídos e o ground truth manual (classes e relações). Os resultados indicam acurácia de 96,4–100\% para classes e 65,0–90,4\% para relações, com desempenho considerado aceitável, variando conforme a complexidade das dependências.

\paragraph{Síntese (RQ6).}
\textit{Desenho:} estudo experimental com 5 sistemas OO.  
\textit{Métricas:} tempo e acurácia (classes e relações).  
\textit{Resultados:} classes 96,4–100\%; relações 65,0–90,4\%.  
\textit{Qualidade prática:} boa precisão estrutural; limitações de escala e avaliação.


\begin{table}[h!]
\centering
\scriptsize % ou \footnotesize, se quiser mais compacta
\setlength{\tabcolsep}{4pt} % espaçamento horizontal entre colunas
\renewcommand{\arraystretch}{1.2} % espaçamento vertical entre linhas

\begin{tabularx}{\textwidth}{Y Y Y Y Y Y Y}
\toprule
\textbf{Autores / Referência} &
\textbf{Linguagem / Domínio} &
\textbf{Modelo Gerado} &
\textbf{Aspecto} &
\textbf{Técnica / Transformação} &
\textbf{Ferramenta / Framework} &
\textbf{Validação / Estudo de Caso} \\
\midrule

\midrule
\textcite{zhang2016j2x} &
Java; pequenos sistemas OO (eLib, Minesweeper, Blog, PayrollSys, myAlgsLib) &
UML Classe; UML Sequência &
Estático — compreensão e redocumentação &
Código$\rightarrow$AST$\rightarrow$J2X; sentenças simplif.$\rightarrow$OFG; CFG+OFG$\rightarrow$Sequência &
J2UML; JavaCC; DOM4J; J2X (DTD/XML) &
5 casos pequenos; acurácia: classes 96,4–100\%; relações 65,0–90,4\% \\

\bottomrule
\end{tabularx}
\caption{Resumo das abordagens}
\label{tab:mdre6}
\end{table}

\newpage

\printbibliography

\end{document}
