\documentclass[12pt,a4paper]{article}

\usepackage[english,portuguese]{babel}
\usepackage[utf8]{inputenc}
\usepackage[T1]{fontenc}
\usepackage{csquotes}

\usepackage[
    backend=biber,
    style=abnt,
    % citestyle=abnt-numeric,
    sorting=none
    ]{biblatex}
    
\addbibresource{refs.bib}

\begin{document}

\section{Colossal Cave Adventure}
O repositório escolhido para análise corresponde a uma reimplementação em Python 3 do jogo Colossal Cave Adventure, um dos primeiros softwares interativos baseados em texto da história da computação. A escolha desse projeto fundamenta-se em três aspectos principais: relevância histórica, clareza estrutural e valor pedagógico para a análise de engenharia de software.

Em primeiro lugar, trata-se de um sistema originalmente desenvolvido na década de 1970 e posteriormente portado para uma linguagem moderna, preservando a lógica e os artefatos conceituais do projeto original. Essa característica torna o repositório um objeto adequado para estudos de engenharia reversa, pois permite comparar como decisões de projeto, organização de código e estruturas de dados evoluíram ao longo do tempo sem alterar o comportamento essencial do sistema.

Em segundo lugar, o projeto apresenta excelente organização e rastreabilidade interna. O código é modularizado, possui testes automatizados que cobrem cenários completos de execução e mantém um arquivo de dados (advent.dat) que descreve integralmente o universo do jogo. Essa estrutura facilita a análise da relação entre requisitos, código e testes, permitindo observar como comportamentos e regras de negócio são formalizados em artefatos de software.

Por fim, o repositório foi escolhido por seu valor didático e reprodutível. Por ser escrito em Python — uma linguagem amplamente utilizada em contextos acadêmicos —, o código é acessível e permite a execução, modificação e instrumentação necessárias às atividades de engenharia de requisitos e análise de rastreabilidade. Além disso, o projeto é licenciado sob Apache 2.0, o que garante liberdade para uso, estudo e reprodução dos experimentos dentro do contexto acadêmico.

Dessa forma, o repositório Colossal Cave Adventure (Python port) constitui um objeto de estudo adequado para o presente trabalho, possibilitando a aplicação de técnicas de engenharia reversa, extração de modelos conceituais e análise de rastreabilidade entre código, testes e especificações funcionais.

O processo de criação do Colossal Cave Adventure por Will Crowther ilustra, de forma histórica e simbólica, a importância da compreensão do domínio do problema na concepção de sistemas. Antes de escrever o código do jogo, Crowther participou ativamente de expedições reais da Cave Research Foundation (CRF) para mapear a Colossal Cave, uma rede de cavernas localizada no parque nacional de Mammoth Cave, em Kentucky. Esse contato direto com o ambiente físico lhe permitiu estruturar o jogo não como uma abstração puramente imaginária, mas como uma representação computacional de um espaço real, com regras, obstáculos e interações inspiradas em experiências concretas (Jerz, 2007).

Ao traduzir esse espaço físico em código, Crowther não apenas programou um jogo, mas modelou um sistema a partir de uma compreensão profunda do domínio — um princípio central da engenharia de software moderna. Sua abordagem reflete a prática de elicitação e análise de requisitos: observar o ambiente, identificar entidades relevantes, compreender as relações entre elas e transformá-las em estruturas e comportamentos formais. Assim como o engenheiro de requisitos interpreta as necessidades e restrições de um sistema real, Crowther interpretou o espaço físico da caverna e suas dinâmicas, convertendo-as em regras lógicas e estruturas de dados.

Essa correspondência entre compreensão empírica e representação computacional demonstra que, mesmo em um contexto histórico anterior à formalização da engenharia de requisitos, já se manifestava a preocupação em alinhar o modelo do sistema à realidade do domínio. Nesse sentido, Adventure pode ser visto não apenas como o primeiro exemplo de ficção interativa, mas também como uma das primeiras demonstrações práticas de modelagem de domínio e interpretação de problema aplicada à programação.

\printbibliography

\end{document}