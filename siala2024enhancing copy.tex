\documentclass[12pt,a4paper]{article}
\usepackage[utf8]{inputenc}
\usepackage[T1]{fontenc}

% Idioma e tipografia
\usepackage[brazil]{babel}
\usepackage{csquotes}
\usepackage{lmodern}
\usepackage{microtype}

% Layout e recursos básicos
\usepackage[left=3cm,right=3cm,top=3cm,bottom=3cm]{geometry}
\usepackage{graphicx}
\usepackage{hyperref}
\usepackage{bookmark}

% Matemática e teoremas (essencial só se você usa)
\usepackage{amsmath,amssymb,amsthm}

% Tabelas em paisagem e colunas flexíveis
% \usepackage{pdflscape}
\usepackage{tabularx}
\usepackage{booktabs}
\usepackage{ragged2e}
\usepackage{array} % para \newcolumntype
\newcolumntype{Y}{>{\RaggedRight\arraybackslash}X}
\usepackage{rotating} % para sidewaystable/sideways


% Bibliografia ABNT numerada
\usepackage[
  backend=biber,
  style=abnt,          % estilo bibliográfico
  sorting=none,
  giveninits=true,
  uniquename=false, 
  doi=false,
  isbn=false,
  url=false,
  language=brazil,
  scbib,
  ittitles,
  justify
]{biblatex}
\addbibresource{refs.bib}

% ======= PADRONIZAÇÃO PARA A TABELA MDRE =======

% Coluna flexível "Y" (se ainda não tiver)
% \usepackage{tabularx,booktabs,ragged2e,array}
% \newcolumntype{Y}{>{\RaggedRight\arraybackslash}X}

% 1) Vocabulário controlado (sempre em SMALL CAPS):
\newcommand{\Static}{\textsc{Estático}}
\newcommand{\Dynamic}{\textsc{Dinâmico}}
\newcommand{\Hybrid}{\textsc{Híbrido}}
\newcommand{\Comp}{\textsc{Compreensão}}
\newcommand{\Redoc}{\textsc{Redocumentação}}
\newcommand{\Mig}{\textsc{Migração}}
\newcommand{\Quali}{\textsc{Qualidade}}

% 2) Macros para setas e encadeamentos:
\newcommand{\ctoa}{\(\text{Código} \rightarrow \text{AST}\)}
\newcommand{\atoxi}{\(\text{AST} \rightarrow \text{IM}\)}   % IM = modelo intermediário
\newcommand{\imtoxml}{\(\text{IM} \rightarrow \text{XML}\)}
\newcommand{\imtomdl}{\(\text{IM} \rightarrow \text{UML}\)}
\newcommand{\tmtomdl}{\(\text{T2M/M2M} \rightarrow \text{UML}\)}
\newcommand{\xtoSeq}{\(\rightarrow \text{UML Sequência}\)}
\newcommand{\xtoClass}{\(\rightarrow \text{UML Classe}\)}
\newcommand{\xtoAct}{\(\rightarrow \text{UML Atividade}\)}

% 3) Abreviações de ferramentas (consistentes):
\newcommand{\EMF}{Eclipse/EMF}
\newcommand{\UMLtwo}{UML2}
\newcommand{\PlantUML}{PlantUML}
\newcommand{\JavaParser}{JavaParser}

% 4) Formato da célula “Aspecto”: Técnica ; Objetivo(s)
%    Ex.: \Static; \Comp/\Redoc (estrutura + comportamento)

% 5) Formato da célula “Técnica/Transformação”:
%    Use sempre cadeia com “→”, negrite elementos-chaves e padronize nomes.
%    Ex.: Código → AST → \textbf{IM(XML)} → T2M/M2M → \textbf{UML2}
%
% 6) Formato da célula “Validação”:
%    [tipo de evidência; dataset/projetos; métrica(s) ou avaliação; nota curta]
%    Ex.: OSS (9 projetos, 2640 classes); AUC=0.73; custo de rótulo 10%

\begin{document}

\begin{titlepage}
    \begin{center}
        \vspace*{0cm}
        
            \includegraphics[width=0.5\textwidth]{Images/Logo_FGV.png} 
            
        \vspace{1.5cm}
        \large
        
        Ciência de Dados e I.A.\\
        Escola de Matemática Aplicada\\
        Fundação Getúlio Vargas\\

        \vspace{1cm}  
    
        \Large
        Engenharia de Requisitos
            
        \vspace{2cm}
        
        \vspace{0.25cm}

        \Huge \textbf{TCC} \\ 
        \vspace{0.5cm}
        \huge \textbf{Enhancing Model-Driven Reverse Engineering Using Machine
Learning}
        \vspace{3.6cm}
        
        \large
                Aluno: Isabela Yabe\\
                Orientador: Rafael de Pinho André\\
                Escola de Matemática Aplicada, FGV/EMAp \\
                Rio de Janeiro - RJ.
        \vfill
            
        \vspace{0.8cm}  
        
        Rio de Janeiro, 2025
            
    \end{center}
\end{titlepage}

\newpage
\pagenumbering{arabic}
\section{Revisão literária}
Artigo revisado \textcite{siala2024enhancing}:

A revisão tem o objetivo de compreender o estado da arte das abordagens de engenharia reversa que partem de código-fonte e produzem artefatos de alto nível, como diagramas UML. Para garantir uma análise sistemática e comparável entre diferentes propostas, foram definidas perguntas de pesquisa (\textit{Research Questions — RQs}) que orientam a coleta e síntese dos dados extraídos dos estudos selecionados.

\begin{itemize}
  \item \textbf{RQ1.} Em quais linguagens e domínios as abordagens que partem de código-fonte foram aplicadas?
  \item \textbf{RQ2.} Quais modelos/artefatos de alto nível são gerados?
  \item \textbf{RQ3.} Qual aspecto é privilegiado (estático, dinâmico, híbrido) e com qual objetivo (compreensão, redocumentação, migração, qualidade)?
  \item \textbf{RQ4.} Quais técnicas e transformações viabilizam a passagem do código para o modelo de alto nível?
  \item \textbf{RQ5.} Quais ferramentas/frameworks são utilizados?
  \item \textbf{RQ6.} Como as abordagens são validadas e com que qualidade prática?
\end{itemize}

\section{RQ1. Em quais linguagens e domínios as abordagens que partem de código-fonte foram aplicadas?}
Java e Python são as linguagens de código-fonte alvo do método MDRE proposto. O artigo situa o problema em sistemas legados corporativos e industriais, caracterizados por alta complexidade e longa vida útil.


\section{RQ2. Quais modelos/artefatos de alto nível são gerados?}
O método MDRE proposto tem como objetivo gerar especificações UML (diagramas de classe) e OCL (restrições, invariantes e regras de negócio) a partir de código-fonte. Esses artefatos representam tanto a estrutura quanto a semântica dos sistemas, permitindo a compreensão e redocumentação automatizada.

\section{RQ3. Qual aspecto é privilegiado (estático, dinâmico, híbrido) e com qual objetivo (compreensão, redocumentação, migração, qualidade)?}
O aspecto privilegiado é estático, e o objetivo principal é compreensão e redocumentação, com foco secundário em migração de sistemas legados. A abordagem propõe combinar análise estrutural com abstração semântica por meio de LLMs, o que amplia o potencial de entendimento do código.

\section{RQ4. Quais técnicas e transformações viabilizam a passagem do código para o modelo de alto nível?}
A proposta estrutura-se como um pipeline MDRE híbrido que integra transformações modelo–modelo (M2M) tradicionais com técnicas de aprendizado de máquina baseadas em LLMs. O fluxo é dividido em quatro etapas principais:

\begin{itemize}
    \item \textbf{Pré-processamento:} tokenização e simplificação semântica do código, reduzindo a complexidade sintática e preparando-o para interpretação pelos LLMs. Essa fase atua como um modelo intermediário (\emph{Text-to-Model}).
    \item \textbf{Codificação com LLMs:} uso de modelos \emph{transformer}-based \emph{encoder-decoder} para traduzir o código em uma representação textual semântica intermediária (pré-UML/OCL), análoga a uma transformação M2M semântica.
    \item \textbf{Pós-processamento e \emph{Model Repair}:} refinamento sintático e correção automática dos modelos gerados, assegurando consistência e completude das restrições OCL e dos elementos UML.
    \item \textbf{Geração de diagramas:} conversão das representações textuais validadas em diagramas formais UML e OCL, utilizando ferramentas como PlantUML, Graphviz e Modelio (fase M2V).
\end{itemize}

O processo adota um aprendizado bidirecional, no qual o modelo aprende a traduzir tanto de código para modelo quanto de modelo para código, reforçando a consistência entre os domínios e aprimorando a capacidade de generalização.

\section{RQ5. Quais ferramentas/frameworks são utilizados?}
A proposta combina o paradigma MDA para interoperabilidade de modelos com técnicas de aprendizado profundo e ferramentas UML consolidadas. O ecossistema integra:

\begin{itemize}
    \item o framework \textbf{OMG/MDA} e a ferramenta \textbf{AgileUML} para padronização e estruturação dos modelos;
    \item modelos de linguagem de grande escala (\textbf{LLMs}) baseados em arquitetura Transformer, responsáveis pela codificação e abstração semântica;
    \item ferramentas de modelagem \textbf{Graphviz}, \textbf{PlantUML} e \textbf{Modelio}, utilizadas para gerar representações visuais UML e OCL a partir das saídas textuais.
\end{itemize}

Essa combinação sustenta um pipeline MDRE automatizado voltado à compreensão, redocumentação e migração de sistemas legados escritos em Java e Python.

\section{RQ6. Como as abordagens são validadas e com que qualidade prática?}
A validação é estruturada dentro do paradigma \emph{Design Science Methodology} (DSM) e ainda está planejada de forma prospectiva. O processo prevê duas etapas principais:  
(i) \textbf{Comparação com baseline}, aplicando o mesmo código-fonte em um método MDRE tradicional e na nova abordagem baseada em LLMs; e  
(ii) \textbf{Estudos de caso} com dois sistemas reais de diferentes tamanhos e domínios.

Os critérios de avaliação incluem: (a) correção semântica e completude dos modelos em relação ao código; e (b) qualidade e compreensibilidade dos diagramas gerados. Os autores propõem medições quantitativas e análise de especialistas, visando mensurar a melhoria na acurácia e na utilidade prática das representações produzidas.

\begin{table}[h!]
\centering
\scriptsize
\setlength{\tabcolsep}{4pt}
\renewcommand{\arraystretch}{1.2}

\begin{tabularx}{\textwidth}{Y Y Y Y Y Y Y}
\toprule
\textbf{Autores / Referência} &
\textbf{Linguagem / Domínio} &
\textbf{Modelo Gerado} &
\textbf{Aspecto} &
\textbf{Técnica / Transformação} &
\textbf{Ferramenta / Framework} &
\textbf{Validação / Estudo de Caso} \\
\midrule

\textcite{siala2024enhancing} &
Java; Python; sistemas legados &
UML Classe; OCL &
Estático — compreensão, redocumentação e migração &
Código $\rightarrow$ tokenização/simplificação $\rightarrow$ geração textual UML/OCL (LLM) $\rightarrow$ model repair $\rightarrow$ diagramas UML/OCL &
Graphviz; PlantUML; Modelio; AgileUML; LLM &
Comparação MDRE: dois estudos de caso; correção semântica, completude e compreensibilidade \\

\bottomrule
\end{tabularx}
\caption{Resumo da abordagem MDRE híbrida baseada em LLMs (\textcite{siala2024enhancing})}
\label{tab:mdre6}
\end{table}


\newpage

\printbibliography

\end{document}
