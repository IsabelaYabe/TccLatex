\documentclass[12pt,a4paper]{article}
\usepackage[utf8]{inputenc}
\usepackage[T1]{fontenc}

% Idioma e tipografia
\usepackage[brazil]{babel}
\usepackage{csquotes}
\usepackage{lmodern}
\usepackage{microtype}

% Layout e recursos básicos
\usepackage[left=3cm,right=3cm,top=3cm,bottom=3cm]{geometry}
\usepackage{graphicx}
\usepackage{hyperref}
\usepackage{bookmark}

% Matemática e teoremas (essencial só se você usa)
\usepackage{amsmath,amssymb,amsthm}

% Tabelas em paisagem e colunas flexíveis
% \usepackage{pdflscape}
\usepackage{tabularx}
\usepackage{booktabs}
\usepackage{ragged2e}
\usepackage{array} % para \newcolumntype
\newcolumntype{Y}{>{\RaggedRight\arraybackslash}X}
\usepackage{rotating} % para sidewaystable/sideways


% Bibliografia ABNT numerada
\usepackage[
  backend=biber,
  style=abnt,          % estilo bibliográfico
  sorting=none,
  giveninits=true,
  uniquename=false, 
  doi=false,
  isbn=false,
  url=false,
  language=brazil,
  scbib,
  ittitles,
  justify
]{biblatex}
\addbibresource{refs.bib}

% ======= PADRONIZAÇÃO PARA A TABELA MDRE =======

% Coluna flexível "Y" (se ainda não tiver)
% \usepackage{tabularx,booktabs,ragged2e,array}
% \newcolumntype{Y}{>{\RaggedRight\arraybackslash}X}

% 1) Vocabulário controlado (sempre em SMALL CAPS):
\newcommand{\Static}{\textsc{Estático}}
\newcommand{\Dynamic}{\textsc{Dinâmico}}
\newcommand{\Hybrid}{\textsc{Híbrido}}
\newcommand{\Comp}{\textsc{Compreensão}}
\newcommand{\Redoc}{\textsc{Redocumentação}}
\newcommand{\Mig}{\textsc{Migração}}
\newcommand{\Quali}{\textsc{Qualidade}}

% 2) Macros para setas e encadeamentos:
\newcommand{\ctoa}{\(\text{Código} \rightarrow \text{AST}\)}
\newcommand{\atoxi}{\(\text{AST} \rightarrow \text{IM}\)}   % IM = modelo intermediário
\newcommand{\imtoxml}{\(\text{IM} \rightarrow \text{XML}\)}
\newcommand{\imtomdl}{\(\text{IM} \rightarrow \text{UML}\)}
\newcommand{\tmtomdl}{\(\text{T2M/M2M} \rightarrow \text{UML}\)}
\newcommand{\xtoSeq}{\(\rightarrow \text{UML Sequência}\)}
\newcommand{\xtoClass}{\(\rightarrow \text{UML Classe}\)}
\newcommand{\xtoAct}{\(\rightarrow \text{UML Atividade}\)}

% 3) Abreviações de ferramentas (consistentes):
\newcommand{\EMF}{Eclipse/EMF}
\newcommand{\UMLtwo}{UML2}
\newcommand{\PlantUML}{PlantUML}
\newcommand{\JavaParser}{JavaParser}

% 4) Formato da célula “Aspecto”: Técnica ; Objetivo(s)
%    Ex.: \Static; \Comp/\Redoc (estrutura + comportamento)

% 5) Formato da célula “Técnica/Transformação”:
%    Use sempre cadeia com “→”, negrite elementos-chaves e padronize nomes.
%    Ex.: Código → AST → \textbf{IM(XML)} → T2M/M2M → \textbf{UML2}
%
% 6) Formato da célula “Validação”:
%    [tipo de evidência; dataset/projetos; métrica(s) ou avaliação; nota curta]
%    Ex.: OSS (9 projetos, 2640 classes); AUC=0.73; custo de rótulo 10%

\begin{document}

\begin{titlepage}
    \begin{center}
        \vspace*{0cm}
        
            \includegraphics[width=0.5\textwidth]{Images/Logo_FGV.png} 
            
        \vspace{1.5cm}
        \large
        
        Ciência de Dados e I.A.\\
        Escola de Matemática Aplicada\\
        Fundação Getúlio Vargas\\

        \vspace{1cm}  
    
        \Large
        Engenharia de Requisitos
            
        \vspace{2cm}
        
        \vspace{0.25cm}

        \Huge \textbf{TCC} \\ 
        \vspace{0.5cm}
        \huge \textbf{WIP: Generating Sequence Diagrams for Modern Fortran}
        \vspace{3.6cm}
        
        \large
                Aluno: Isabela Yabe\\
                Orientador: Rafael de Pinho André\\
                Escola de Matemática Aplicada, FGV/EMAp \\
                Rio de Janeiro - RJ.
        \vfill
            
        \vspace{0.8cm}  
        
        Rio de Janeiro, 2025
            
    \end{center}
\end{titlepage}
\newpage
\pagenumbering{roman}
% \tableofcontents

\newpage
\pagenumbering{arabic}

\section{Revisão literária}
Artigo revisado \textcite{leatongkam2017generating}:

A revisão tem o objetivo de compreender o estado da arte das abordagens de engenharia reversa que partem de código-fonte e produzem artefatos de alto nível, como diagramas UML. Para garantir uma análise sistemática e comparável entre diferentes propostas, foram definidas perguntas de pesquisa (\textit{Research Questions — RQs}) que orientam a coleta e síntese dos dados extraídos dos estudos selecionados.

\begin{itemize}
  \item \textbf{RQ1.} Em quais linguagens e domínios as abordagens que partem de código-fonte foram aplicadas?
  \item \textbf{RQ2.} Quais modelos/artefatos de alto nível são gerados?
  \item \textbf{RQ3.} Qual aspecto é privilegiado (estático, dinâmico, híbrido) e com qual objetivo (compreensão, redocumentação, migração, qualidade)?
  \item \textbf{RQ4.} Quais técnicas e transformações viabilizam a passagem do código para o modelo de alto nível?
  \item \textbf{RQ5.} Quais ferramentas/frameworks são utilizados?
  \item \textbf{RQ6.} Como as abordagens são validadas e com que qualidade prática?
\end{itemize}

\section{RQ1. Em quais linguagens e domínios as abordagens que partem de código-fonte foram aplicadas?}

A abordagem foi aplicada à linguagem Fortran moderna (OO Fortran).
  
O domínio de aplicação é o da computação científica.

\section{RQ2. Quais modelos/artefatos de alto nível são gerados?}
Os modelos e artefatos de alto nível gerados são derivados da engenharia reversa de código Fortran orientado a objetos, produzindo representações UML que capturam tanto a estrutura quanto o comportamento do sistema.

Originalmente implementado pela ferramenta ForUML, que extrai de código Fortran OO as estruturas de classes, atributos, métodos e relações de agregação e herança.

Esses diagramas representam o aspecto estático do sistema, isto é, sua organização estrutural e hierarquias de classes.

O trabalho propõe uma extensão da ferramenta ForUML para gerar também diagramas de sequência.

O processo de extração gera um arquivo intermediário em formato XMI, que armazena a estrutura do diagrama UML para importação em ferramentas de modelagem como o ArgoUML.

\section{RQ3. Qual aspecto é privilegiado (estático, dinâmico, híbrido) e com qual objetivo (compreensão, redocumentação, migração, qualidade)?}

A proposta adota uma abordagem estática, fundamentada na análise do código-fonte Fortran. Concentrando-se exclusivamente na inspeção estrutural e semântica do código.

O propósito declarado pelos autores é auxiliar a compreensão e redocumentação de programas científicos escritos em Fortran OO, cujo ciclo de vida é longo e frequentemente carece de documentação de design.

\section{RQ4. Quais técnicas e transformações viabilizam a passagem do código para o modelo de alto nível?}

Se baseiam em três etapas principais: definição de regras, extração e geração de modelo.

A primeira etapa consiste em definir regras formais de mapeamento entre elementos da linguagem Fortran OO e notações de diagramas UML de sequência. Ou seja, o sistema identifica padrões sintáticos no código (como chamadas de método, criação de objetos e controle de fluxo) e os converte em elementos comportamentais UML.

NA segunda etapa um módulo de extração percorre o código-fonte Fortran e constrói uma árvore de nós (tree node structure) que representa a hierarquia de classes e suas relações. Durante essa travessia, o código é validado quanto à gramática Fortran, garantindo consistência sintática antes da geração dos diagramas. Essa estrutura intermediária atua como um modelo abstrato do sistema, análogo a um Abstract Syntax Tree (AST) estendida com semântica de classes e interações.

Por fim, na terceira etapa o processo de transformação produz um modelo intermediário XMI, padrão utilizado para representar diagramas UML em formato intercambiável. Esse arquivo é então importado por ferramentas de modelagem como o ArgoUML, que renderiza os diagramas de sequência resultantes.



\section{RQ4. Quais técnicas e transformações viabilizam a passagem do código para o modelo de alto nível?}

O processo se baseia em três etapas principais: definição de regras de mapeamento, extração estrutural e geração do modelo UML.

Na primeira etapa, são definidas regras formais que associam elementos da linguagem Fortran OO a notações de diagramas UML. O sistema identifica padrões sintáticos, como chamadas de método, criação de objetos e estruturas de controle, e os converte em elementos comportamentais UML.

Na segunda etapa, um módulo de extração percorre o código-fonte e constrói uma estrutura em árvore de nós (tree node structure) que representa classes, métodos e relações. Durante essa travessia, o código é validado segundo a gramática Fortran, assegurando consistência sintática. Essa estrutura atua como um modelo abstrato do sistema, análogo a uma AST (Abstract Syntax Tree) enriquecida com semântica de classes e interações.

Por fim, a terceira etapa gera um modelo intermediário XMI, padrão da OMG para representação de diagramas UML. Esse arquivo é então importado por ferramentas como o ArgoUML, que renderiza automaticamente os diagramas de sequência.


\section{RQ5. Quais ferramentas/frameworks são utilizados?}

as ferramentas e frameworks empregados concentram-se na extensão da infraestrutura já existente da ferramenta ForUML (Fortran UML Reverse Engineering Tool), complementada por tecnologias padrão de modelagem UML e interoperabilidade via XMI.

O ForUML é a ferramenta-base do estudo, ela é capas de analisar código-fonte Fortran orientado a objetos e gerar diagramas UML de classes. E será agora estendida para incluir diagramas de sequência.

O XMI é o formato de intercâmbio padronizado pela OMG (Object Management Group), responsável por definir as especificações UML utilizadas no mapeamento.

Ferramenta open-source usada para visualização dos modelos UML gerados pela ForUML é o ArgoUML é compatível com o padrão XMI (XML Metadata Interchange), o que permite importar o modelo intermediário e renderizar automaticamente o diagrama.

As regras de transformação entre código Fortran e notação UML seguem o padrão de especificação UML do OMG. Isso garante conformidade semântica e sintática com o metamodelo UML e integração com outras ferramentas compatíveis.


\section{RQ5. Quais ferramentas/frameworks são utilizados?}

As ferramentas empregadas concentram-se na extensão da infraestrutura da ForUML (Fortran UML Reverse Engineering Tool), complementada por tecnologias padrão de modelagem UML e interoperabilidade via XMI.

A ForUML é a ferramenta-base do estudo: capaz de analisar código-fonte Fortran OO e gerar diagramas UML de classes, sendo agora expandida para incluir diagramas de sequência.

O XMI (XML Metadata Interchange) é o formato de intercâmbio padronizado pela OMG (Object Management Group), utilizado para garantir a compatibilidade e a conformidade com o metamodelo UML.

A visualização dos diagramas é realizada no ArgoUML, ferramenta open-source compatível com o padrão XMI, que permite importar e renderizar automaticamente os modelos gerados.

As regras de transformação seguem as especificações da UML 2.x da OMG, assegurando aderência semântica e integração com outros frameworks de modelagem.

\section{RQ6. Como as abordagens são validadas e com que qualidade prática?}
O estudo é apresentado como um work in progress, descrevendo as etapas conceituais e o plano de testes para a nova funcionalidade de geração de diagramas de sequência no ForUML.

O artigo menciona que o projeto foi publicado no GitHub para incentivar o uso e colaboração da comunidade científica. 

\begin{table}[h!]
\centering
\scriptsize % ou \footnotesize, se quiser mais compacta
\setlength{\tabcolsep}{4pt} % espaçamento horizontal entre colunas
\renewcommand{\arraystretch}{1.2} % espaçamento vertical entre linhas

\begin{tabularx}{\textwidth}{Y Y Y Y Y Y Y}
\toprule
\textbf{Autores / Referência} &
\textbf{Linguagem / Domínio} &
\textbf{Modelo Gerado} &
\textbf{Aspecto} &
\textbf{Técnica / Transformação} &
\textbf{Ferramenta / Framework} &
\textbf{Validação / Estudo de Caso} \\
\midrule

\textcite{leatongkam2017generating} &
Fortran OO; computação científica e engenharia &
UML Classe; UML Sequência; modelo intermediário XMI &
Estático — compreensão e redocumentação &
Regras de mapeamento código $\rightarrow$ UML (OMG); parsing estático; árvore sintática; geração XMI $\rightarrow$ importação ArgoUML &
ForUML (extensão); ArgoUML; padrão OMG UML/XMI;&
\textit{Work in progress} \\

\bottomrule
\end{tabularx}
\caption{Resumo das abordagens}
\label{tab:mdre6}
\end{table}

\newpage

\printbibliography

\end{document}
