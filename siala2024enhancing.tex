\documentclass[12pt,a4paper]{article}
\usepackage[utf8]{inputenc}
\usepackage[T1]{fontenc}

% Idioma e tipografia
\usepackage[brazil]{babel}
\usepackage{csquotes}
\usepackage{lmodern}
\usepackage{microtype}

% Layout e recursos básicos
\usepackage[left=3cm,right=3cm,top=3cm,bottom=3cm]{geometry}
\usepackage{graphicx}
\usepackage{hyperref}
\usepackage{bookmark}

% Matemática e teoremas (essencial só se você usa)
\usepackage{amsmath,amssymb,amsthm}

% Tabelas em paisagem e colunas flexíveis
% \usepackage{pdflscape}
\usepackage{tabularx}
\usepackage{booktabs}
\usepackage{ragged2e}
\usepackage{array} % para \newcolumntype
\newcolumntype{Y}{>{\RaggedRight\arraybackslash}X}
\usepackage{rotating} % para sidewaystable/sideways


% Bibliografia ABNT numerada
\usepackage[
  backend=biber,
  style=abnt,          % estilo bibliográfico
  sorting=none,
  giveninits=true,
  uniquename=false, 
  doi=false,
  isbn=false,
  url=false,
  language=brazil,
  scbib,
  ittitles,
  justify
]{biblatex}
\addbibresource{refs.bib}

% ======= PADRONIZAÇÃO PARA A TABELA MDRE =======

% Coluna flexível "Y" (se ainda não tiver)
% \usepackage{tabularx,booktabs,ragged2e,array}
% \newcolumntype{Y}{>{\RaggedRight\arraybackslash}X}

% 1) Vocabulário controlado (sempre em SMALL CAPS):
\newcommand{\Static}{\textsc{Estático}}
\newcommand{\Dynamic}{\textsc{Dinâmico}}
\newcommand{\Hybrid}{\textsc{Híbrido}}
\newcommand{\Comp}{\textsc{Compreensão}}
\newcommand{\Redoc}{\textsc{Redocumentação}}
\newcommand{\Mig}{\textsc{Migração}}
\newcommand{\Quali}{\textsc{Qualidade}}

% 2) Macros para setas e encadeamentos:
\newcommand{\ctoa}{\(\text{Código} \rightarrow \text{AST}\)}
\newcommand{\atoxi}{\(\text{AST} \rightarrow \text{IM}\)}   % IM = modelo intermediário
\newcommand{\imtoxml}{\(\text{IM} \rightarrow \text{XML}\)}
\newcommand{\imtomdl}{\(\text{IM} \rightarrow \text{UML}\)}
\newcommand{\tmtomdl}{\(\text{T2M/M2M} \rightarrow \text{UML}\)}
\newcommand{\xtoSeq}{\(\rightarrow \text{UML Sequência}\)}
\newcommand{\xtoClass}{\(\rightarrow \text{UML Classe}\)}
\newcommand{\xtoAct}{\(\rightarrow \text{UML Atividade}\)}

% 3) Abreviações de ferramentas (consistentes):
\newcommand{\EMF}{Eclipse/EMF}
\newcommand{\UMLtwo}{UML2}
\newcommand{\PlantUML}{PlantUML}
\newcommand{\JavaParser}{JavaParser}

% 4) Formato da célula “Aspecto”: Técnica ; Objetivo(s)
%    Ex.: \Static; \Comp/\Redoc (estrutura + comportamento)

% 5) Formato da célula “Técnica/Transformação”:
%    Use sempre cadeia com “→”, negrite elementos-chaves e padronize nomes.
%    Ex.: Código → AST → \textbf{IM(XML)} → T2M/M2M → \textbf{UML2}
%
% 6) Formato da célula “Validação”:
%    [tipo de evidência; dataset/projetos; métrica(s) ou avaliação; nota curta]
%    Ex.: OSS (9 projetos, 2640 classes); AUC=0.73; custo de rótulo 10%

\begin{document}

\begin{titlepage}
    \begin{center}
        \vspace*{0cm}
        
            \includegraphics[width=0.5\textwidth]{Images/Logo_FGV.png} 
            
        \vspace{1.5cm}
        \large
        
        Ciência de Dados e I.A.\\
        Escola de Matemática Aplicada\\
        Fundação Getúlio Vargas\\

        \vspace{1cm}  
    
        \Large
        Engenharia de Requisitos
            
        \vspace{2cm}
        
        \vspace{0.25cm}

        \Huge \textbf{TCC} \\ 
        \vspace{0.5cm}
        \huge \textbf{Enhancing Model-Driven Reverse Engineering Using Machine
Learning}
        \vspace{3.6cm}
        
        \large
                Aluno: Isabela Yabe\\
                Orientador: Rafael de Pinho André\\
                Escola de Matemática Aplicada, FGV/EMAp \\
                Rio de Janeiro - RJ.
        \vfill
            
        \vspace{0.8cm}  
        
        Rio de Janeiro, 2025
            
    \end{center}
\end{titlepage}
\newpage
\pagenumbering{roman}
\tableofcontents

\newpage
\pagenumbering{arabic}

\section{Revisão literária}
Artigo revisado \textcite{siala2024enhancing}:

A revisão tem o objetivo de compreender o estado da arte das abordagens de engenharia reversa que partem de código-fonte e produzem artefatos de alto nível, como diagramas UML. Para garantir uma análise sistemática e comparável entre diferentes propostas, foram definidas perguntas de pesquisa (\textit{Research Questions — RQs}) que orientam a coleta e síntese dos dados extraídos dos estudos selecionados.

\begin{itemize}
  \item \textbf{RQ1.} Em quais linguagens e domínios as abordagens que partem de código-fonte foram aplicadas?
  \item \textbf{RQ2.} Quais modelos/artefatos de alto nível são gerados?
  \item \textbf{RQ3.} Qual aspecto é privilegiado (estático, dinâmico, híbrido) e com qual objetivo (compreensão, redocumentação, migração, qualidade)?
  \item \textbf{RQ4.} Quais técnicas e transformações viabilizam a passagem do código para o modelo de alto nível?
  \item \textbf{RQ5.} Quais ferramentas/frameworks são utilizados?
  \item \textbf{RQ6.} Como as abordagens são validadas e com que qualidade prática?
\end{itemize}

\section{Em quais linguagens e domínios as abordagens que partem de código-fonte foram aplicadas?}
Java e Python são as linguagens de código-fonte alvo do método MDRE proposto. O artigo situa o problema em sistemas legados corporativos e industriais, caracterizados por alta complexidade e longa vida útil.

\section{Quais modelos/artefatos de alto nível são gerados?}

O método MDRE proposto visa gerar especificações UML (diagramas de classe) e OCL(restrições, invariantes e regras de negócio) a partir do código-fonte.

\section{Qual aspecto é privilegiado (estático, dinâmico, híbrido) e com qual objetivo (compreensão, redocumentação, migração, qualidade)?}

O aspecto privilegiado é estático, e o objetivo principal é compreensão e redocumentação, com foco secundário em migração de sistemas legados.

\section{Quais técnicas e transformações viabilizam a passagem do código para o modelo de alto nível?}

Segundo \textcite{siala2024enhancing}, a pesquisa busca utilizar LLMs para abstrair especificações UML/OCL a partir de código-fonte, questionando também quais informações devem ser coletadas por meio desses modelos para facilitar tal abstração e de que forma.

artigo, as técnicas e transformações que viabilizam a passagem do código-fonte para modelos UML/OCL de alto nível são estruturadas em um pipeline MDRE híbrido que combina transformações modelo–modelo (M2M) com técnicas de aprendizado de máquina (LLMs).

a transformação ocorre por meio de quatro etapas principais, cada uma correspondendo a um tipo de técnica aplicada:
\begin{itemize}
    \item Pré-processamento: a tokenização e a simplificação semântica de código, prepara o código para interpretação por LLMs, reduzindo a complexidade sintática, atuando como um modelo intermediário (T2M).
    \item Codificação com LLMs: as técnicas de transformer-based encoder-decoder traduz o código em uma representação textual semântica (pré-UML/OCL). Análoga a uma transformação M2M semântica, onde o modelo intermediário é gerado a partir da codificação contextual feita pelo LLM.
    \item Pós-processamento e Model Repair: a correção automática e o refinamento sintático dos modelos gerados, faz garantir a consistência e completude das restrições OCL e dos elementos UML. Aplicando correções na representação textual.
    \item Geração de diagramas: transformação de texto para grafo UML (via PlantUML, Graphviz, Modelio). Representando a transformação M2V.
Função: converter as representações textuais validadas em diagramas formais.
\end{itemize}

O artigo propõe um processo de aprendizado bidirecional, em que o modelo aprende tanto a traduzir de código para modelo, quanto de modelo para código, reforçando a consistência entre os domínios.

O método emprega uma pipeline de transformações híbridas, em que técnicas de análise estática, transformações modelo–modelo (MDA) e LLMs. 

\section{Quais ferramentas/frameworks são utilizados?}
O artigo prevê a geração automática de diagramas UML (principalmente de classes) a partir das saídas textuais do modelo, o artigo cita: Graphviz, PlantUML, Modelio.

Utiliza o frameworks de modelagem e padronização (OMG/MDA, AgileUML) e incorpora LLMs para aprendizado semântico. 

O trabalho de Siala (2024) propõe um ecossistema integrado de ferramentas e frameworks que combinam:
\begin{itemize}
    \item o paradigma MDA (para estrutura e interoperabilidade de modelos);
    \item LLMs baseados em Transformer (para tradução semântica de código),
    \item e ferramentas UML consagradas (Graphviz, PlantUML, Modelio, AgileUML) para gerar automaticamente modelos UML e OCL a partir de código-fonte Java e Python, com vistas à compreensão, redocumentação e migração de sistemas legados.
\end{itemize}

\section{Como as abordagens são validadas e com que qualidade prática?}

A validação da abordagem é planejada de forma estruturada dentro do paradigma de Design Science Methodology (DSM) e envolve comparação empírica, estudos de caso e critérios objetivos de qualidade de modelo.

A validação prática será realizada em duas etapas: (i) Comparação com baseline: aplicar a mesma entrada (código-fonte) em um método MDRE tradicional e na nova abordagem baseada em LLM; (ii) Estudos de caso: dois sistemas de software reais, de diferentes tamanhos e domínios.

Os critérios de avaliação para engenharia reversa incluem (i) a correção semântica e
a completude dos modelos em comparação com o código; (ii) a qualidade e a
compreensibilidade dos diagramas.

\begin{table}[h!]
\centering
\scriptsize % ou \footnotesize, se quiser mais compacta
\setlength{\tabcolsep}{4pt} % espaçamento horizontal entre colunas
\renewcommand{\arraystretch}{1.2} % espaçamento vertical entre linhas

\begin{tabularx}{\textwidth}{Y Y Y Y Y Y Y}
\toprule
\textbf{Autores / Referência} &
\textbf{Linguagem / Domínio} &
\textbf{Modelo Gerado} &
\textbf{Aspecto} &
\textbf{Técnica / Transformação} &
\textbf{Ferramenta / Framework} &
\textbf{Validação / Estudo de Caso} \\
\midrule


\textcite{siala2024enhancing} &
Java; Python; sistemas legados &
UML Classe; OCL &
Estático — compreensão, redocumentação e migração &
código $\rightarrow$ tokenização/simplificação $\rightarrow$ geração textual UML/OCL intermediária(LLM) $\rightarrow$ model repair $\rightarrow$ diagramas UML/OCL &
Graphviz; PlantUML; Modelio; AgileUML; LLM &
Comparação MDRE: dois estudos de caso; correção semântica, completude e compreensibilidade \\


\bottomrule
\end{tabularx}
\caption{Resumo das abordagens MDRE baseadas em código-fonte}
\label{tab:mdre6}
\end{table}

\newpage

\printbibliography

\end{document}
