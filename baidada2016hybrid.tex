\documentclass[12pt,a4paper]{article}
\usepackage[utf8]{inputenc}
\usepackage[T1]{fontenc}

% Idioma e tipografia
\usepackage[brazil]{babel}
\usepackage{csquotes}
\usepackage{lmodern}
\usepackage{microtype}

% Layout e recursos básicos
\usepackage[left=3cm,right=3cm,top=3cm,bottom=3cm]{geometry}
\usepackage{graphicx}
\usepackage{hyperref}
\usepackage{bookmark}

% Matemática e teoremas (essencial só se você usa)
\usepackage{amsmath,amssymb,amsthm}

% Tabelas em paisagem e colunas flexíveis
% \usepackage{pdflscape}
\usepackage{tabularx}
\usepackage{booktabs}
\usepackage{ragged2e}
\usepackage{array} % para \newcolumntype
\newcolumntype{Y}{>{\RaggedRight\arraybackslash}X}
\usepackage{rotating} % para sidewaystable/sideways


% Bibliografia ABNT numerada
\usepackage[
  backend=biber,
  style=abnt,          % estilo bibliográfico
  sorting=none,
  giveninits=true,
  uniquename=false, 
  doi=false,
  isbn=false,
  url=false,
  language=brazil,
  scbib,
  ittitles,
  justify
]{biblatex}
\addbibresource{refs.bib}

% ======= PADRONIZAÇÃO PARA A TABELA MDRE =======

% Coluna flexível "Y" (se ainda não tiver)
% \usepackage{tabularx,booktabs,ragged2e,array}
% \newcolumntype{Y}{>{\RaggedRight\arraybackslash}X}

% 1) Vocabulário controlado (sempre em SMALL CAPS):
\newcommand{\Static}{\textsc{Estático}}
\newcommand{\Dynamic}{\textsc{Dinâmico}}
\newcommand{\Hybrid}{\textsc{Híbrido}}
\newcommand{\Comp}{\textsc{Compreensão}}
\newcommand{\Redoc}{\textsc{Redocumentação}}
\newcommand{\Mig}{\textsc{Migração}}
\newcommand{\Quali}{\textsc{Qualidade}}

% 2) Macros para setas e encadeamentos:
\newcommand{\ctoa}{\(\text{Código} \rightarrow \text{AST}\)}
\newcommand{\atoxi}{\(\text{AST} \rightarrow \text{IM}\)}   % IM = modelo intermediário
\newcommand{\imtoxml}{\(\text{IM} \rightarrow \text{XML}\)}
\newcommand{\imtomdl}{\(\text{IM} \rightarrow \text{UML}\)}
\newcommand{\tmtomdl}{\(\text{T2M/M2M} \rightarrow \text{UML}\)}
\newcommand{\xtoSeq}{\(\rightarrow \text{UML Sequência}\)}
\newcommand{\xtoClass}{\(\rightarrow \text{UML Classe}\)}
\newcommand{\xtoAct}{\(\rightarrow \text{UML Atividade}\)}

% 3) Abreviações de ferramentas (consistentes):
\newcommand{\EMF}{Eclipse/EMF}
\newcommand{\UMLtwo}{UML2}
\newcommand{\PlantUML}{PlantUML}
\newcommand{\JavaParser}{JavaParser}

% 4) Formato da célula “Aspecto”: Técnica ; Objetivo(s)
%    Ex.: \Static; \Comp/\Redoc (estrutura + comportamento)

% 5) Formato da célula “Técnica/Transformação”:
%    Use sempre cadeia com “→”, negrite elementos-chaves e padronize nomes.
%    Ex.: Código → AST → \textbf{IM(XML)} → T2M/M2M → \textbf{UML2}
%
% 6) Formato da célula “Validação”:
%    [tipo de evidência; dataset/projetos; métrica(s) ou avaliação; nota curta]
%    Ex.: OSS (9 projetos, 2640 classes); AUC=0.73; custo de rótulo 10%

\begin{document}

\begin{titlepage}
    \begin{center}
        \vspace*{0cm}
        
            \includegraphics[width=0.5\textwidth]{Images/Logo_FGV.png} 
            
        \vspace{1.5cm}
        \large
        
        Ciência de Dados e I.A.\\
        Escola de Matemática Aplicada\\
        Fundação Getúlio Vargas\\

        \vspace{1cm}  
    
        \Large
        Engenharia de Requisitos
            
        \vspace{2cm}
        
        \vspace{0.25cm}

        \Huge \textbf{TCC} \\ 
        \vspace{0.5cm}
        \huge \textbf{Towards New Hybrid Approach of the Reverse Engineering of UML Sequence Diagram}
        \vspace{3.6cm}
        
        \large
                Aluno: Isabela Yabe\\
                Orientador: Rafael de Pinho André\\
                Escola de Matemática Aplicada, FGV/EMAp \\
                Rio de Janeiro - RJ.
        \vfill
            
        \vspace{0.8cm}  
        
        Rio de Janeiro, 2025
            
    \end{center}
\end{titlepage}
\newpage
\pagenumbering{roman}
% \tableofcontents

\newpage
\pagenumbering{arabic}

\section{Revisão literária}
Artigo revisado \textcite{baidada2016hybrid}:

A revisão tem o objetivo de compreender o estado da arte das abordagens de engenharia reversa que partem de código-fonte e produzem artefatos de alto nível, como diagramas UML. Para garantir uma análise sistemática e comparável entre diferentes propostas, foram definidas perguntas de pesquisa (\textit{Research Questions — RQs}) que orientam a coleta e síntese dos dados extraídos dos estudos selecionados.

\begin{itemize}
  \item \textbf{RQ1.} Em quais linguagens e domínios as abordagens que partem de código-fonte foram aplicadas?
  \item \textbf{RQ2.} Quais modelos/artefatos de alto nível são gerados?
  \item \textbf{RQ3.} Qual aspecto é privilegiado (estático, dinâmico, híbrido) e com qual objetivo (compreensão, redocumentação, migração, qualidade)?
  \item \textbf{RQ4.} Quais técnicas e transformações viabilizam a passagem do código para o modelo de alto nível?
  \item \textbf{RQ5.} Quais ferramentas/frameworks são utilizados?
  \item \textbf{RQ6.} Como as abordagens são validadas e com que qualidade prática?
\end{itemize}

\section{RQ1. Em quais linguagens e domínios as abordagens que partem de código-fonte foram aplicadas?}

A abordagem é declaradamente voltada a linguagens Java. As evidências incluem (i) um exemplo de \textit{control flow graph} (CFG) de um método Java e (ii) a geração de traços por execução em máquina virtual para programas Java, como explicitado pelos autores. 

\section{RQ2. Quais modelos/artefatos de alto nível são gerados?}

O trabalho gera Diagramas de Sequência UML de alto nível (HLSD), com uso uso de operadores de interação (seq, alt, opt, loop, par). 

Os HLSD são construídos com os operadores de combined SD do UML 2: seq, alt, opt, loop, par.


\section{RQ3. Qual aspecto é privilegiado (estático, dinâmico, híbrido) e com qual objetivo (compreensão, redocumentação, migração, qualidade)?}

É proposto uma abordagem híbrida (estático + dinâmico).

O objetivo principal é compreensão e redocumentação, diagramas de sequencia, com efeito direto em manutenção/evolução.

\section{RQ4. Quais técnicas e transformações viabilizam a passagem do código para o modelo de alto nível?}

A abordagem propõe um \textit{pipeline} híbrido composto por quatro etapas articuladas por uma representação intermediária (IR) comportamental baseada em \textit{Colored Petri Nets} (CPN), culminando na geração do \textit{UML Sequence Diagram}.

Na primeira etapa, realiza-se a análise estática do código-fonte por meio da construção de um Control Flow Graph (CFG), que permite identificar condições de execução e derivar valores de entrada capazes de cobrir diferentes comportamentos do sistema.

Em seguida, essas entradas são utilizadas para executar o sistema e coletar traços de execução (\textit{execution traces}) por instrumentação, máquina virtual (no caso de Java) ou depurador customizado. Diversas execuções são realizadas para capturar variações comportamentais, seguidas por uma filtragem de ruído que remove eventos irrelevantes.

Os traços resultantes são convertidos em uma representação intermediária comportamental, estruturada como uma rede de Petri colorida (CPN). Nessa representação, lugares correspondem a diagramas de sequência básicos e transições a operadores de controle (\textit{alt}, \textit{loop}, \textit{seq}, \textit{par}). Um algoritmo incremental percorre os traços linha a linha para construir a CPN e identificar paralelismos por contagem de \textit{threads}.

Por fim, aplica-se a transformação de CPN para UML Sequence Diagram, guiada por regras de mapeamento que traduzem operadores e fluxos em \textit{combined fragments} na notação UML 2.x.

\section{RQ5. Quais ferramentas/frameworks são utilizados?}

O padrão-alvo é a UML 2.x (OMG), utilizada como linguagem de modelagem de destino.
A coleta de traços de execução é realizada por meio de instrumentação do código, uso da máquina virtual Java (JVM) ou de um depurador customizado, dependendo do ambiente analisado.

A representação intermediária é construída sobre o formalismo de Colored Petri Nets (CPN), o que sugere o uso de ferramentas da família CPN Tools para manipulação, análise e simulação das redes.

A transformação final para o Diagrama de Sequência UML é conduzida por regras de mapeamento baseadas em modelo, alinhadas às especificações da UML 2.x, possivelmente implementadas sobre um framework de transformação como o Eclipse Modeling Framework (EMF) ou ferramenta equivalente.

\section{RQ6. Como as abordagens são validadas e com que qualidade prática?}

Não há validação reportada ou experimento, estudo de caso, métrica, análise de desempenho ou comparação sistemática.

\begin{table}[h!]
\centering
\scriptsize % ou \footnotesize, se quiser mais compacta
\setlength{\tabcolsep}{4pt} % espaçamento horizontal entre colunas
\renewcommand{\arraystretch}{1.2} % espaçamento vertical entre linhas

\begin{tabularx}{\textwidth}{Y Y Y Y Y Y Y}
\toprule
\textbf{Autores / Referência} &
\textbf{Linguagem / Domínio} &
\textbf{Modelo Gerado} &
\textbf{Aspecto} &
\textbf{Técnica / Transformação} &
\textbf{Ferramenta / Framework} &
\textbf{Validação / Estudo de Caso} \\
\midrule


\textcite{baidada2016hybrid} &
Java/Genérico; aplicações OO &
UML Sequência (HLSD) &
\textbf{Híbrido} (Estático + Dinâmico); Compreensão/Redocumentação (comportamento) &
CFG$\!\rightarrow$entradas; execuções+traços (filtro); traços$\!\rightarrow$CPN; CPN$\!\rightarrow$UML SD &
Sem ferramenta nominal; UML 2.x; instrumentação/VM/debugger; CPN (IR) &
Sem validação; futuro \\


\bottomrule
\end{tabularx}
\caption{Resumo das abordagens}
\label{tab:mdre6}
\end{table}

\newpage

\printbibliography

\end{document}
